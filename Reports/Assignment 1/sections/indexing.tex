%!TEX root = ../report.tex

\section{Indexing}
\label{sec:indexing}

Nicolai notes on parsing
% Character.isLetter() is damn fast
% http://stackoverflow.com/questions/93976/how-to-determine-whether-a-character-is-a-letter-in-java
- Talk about hyphens and how we handle them. The book says that it's all about finding a heuristic for how to handle them
Handling of hyphens have been inspired by: https://www.ibm.com/developerworks/community/blogs/nlp/entry/tokenization?lang=en

- We automatically exclude terms smaller than three chars
- Don't allow numbers
- Allow terms that start with letters and have numbers in them


Explain your inverted index implementation. As part of your explanation you need to clearly describe how you:
• gather term information while your index program is parsing the collection
• tokenise terms, including how you deal with punctuation and markup tags, and
handle acronyms and hyphenated words
• merge the information together for the final lexicon and invlists files
This explanation should be around a page in length, but no longer than one and a half pages.


