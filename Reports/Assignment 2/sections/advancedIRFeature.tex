%!TEX root = ../report.tex

\section{Advanced IR Feature}
As our advanced information retrieval feature we selected automatic query expansion, also known as pseudo-relevance feedback. Queries might be hard to specify for users and there is a danger that the user searches for a synonym to a word that is more significant in the document collection\,\cite{billerbeckzobel04}. The idea is to expand the initial query with $E$ additional terms that are statistically related to the query to mitigate this vocabulary mismatch\,\cite{billerbeckzobel04}. The steps to automatic query expansion are\,\cite{scholer13}: 

\begin{enumerate}
	\item Perform ranked retrieval on the initial query with a good similarity measure and assume that the top $R$ ranked documents are relevant.
	\item Parse through these $R$ documents and mark all terms in these candidate terms for query expansion.
	\item Select the best $E$ of these candidate terms for the query expansion by evaluating them with some statistical method.
	\item Append the $E$ terms to the initial query and run the ranked retrieval procedure again. This is the final result.
\end{enumerate}

As the statistical method in step three, we use the Okapi Term Selection Value (TSV) approach to select a set of $E$ terms to extend the initial query with\,\cite{billerbeckzobel04}.

\subsection{Implementation}



In your report file, create a heading “Advanced IR Feature”. In this section, you should explain:
• Your chosen advanced IR feature. This must include a clear description of what you are aiming to achieve. You must include a reference to at least one research paper that you have referred to as part of your design.
• A detailed explanation of how you implemented the feature. This section should be no longer than one and a half pages.